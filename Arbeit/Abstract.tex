\chapter*{Kurzfassung}

Beaterkennungsalgorithmen sind Algorithmen,
	die den regelmä{\ss}igen Puls von Musikstücken erkennen,
	zu dem wir Menschen oft beim Hören von Musik unterbewusst mit dem Fu{\ss} oder einem Finger mittippen.
Viele dieser Algorithmen geben auch Schätzungen des Tempos des Musikstücks aus.
In dieser Arbeit wurden die drei Algorithmen
	von~\cite{2001_BeatThis}, \cite{2009_DaPlSt} und \cite{2011_PlRoSt}
	im Echtzeitbetrieb getestet und verglichen.
Dabei wurden sie auf
	Genauigkeit der Tempovorhersagen, Genauigkeit der Beatzeitpunkte, längste korrekte Beatfolge und benötigte Rechenzeit
	untersucht.
Es wurden Tests implementiert,
	die die Algorithmen auf mit Beatzeitpunkten annotieren Liedern eines Datensatzes laufen lassen
	und dabei deren Ausgaben aufzeichnen.
Die Auswertung dieser Aufzeichnungen hat ergeben,
	dass \cite{2009_DaPlSt} in allen vier Tests am besten abgeschnitten hat.
