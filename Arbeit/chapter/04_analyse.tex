\chapter{Analyse der Algorithmen}
\label{analyse}

\section{Selektion}
{
}

\section{Vergleich}
{
}

\section{Funktionsweise}
{
	% TODO: sagen, dass ich hier meine eigenen Implementierung beschreibe, welche aber dem Paper entsprechen sollte und wenn nicht schreib ichs dazu
	% TODO: vielleicht ein Wort zur Notation verlieren wegen Subscript und Klammern
	% TODO: irgendwo erklären was ne ODF ist

	\subsection{2011 Plumbley, Robertson, Stark - Real-Time Visual Beat Tracking Using a Comb Filter Matrix}
	{
		\subsubsection*{Ein- und Ausgabe}
		{
			Der Algorithmus von \cite{2011_PlRoSt} bekommt als Eingabe ein rohes, PCM-kodiertes Audiosignal mit einer Abtastfrequenz von \SI{44.1}{\kilo\hertz}.
			Die Samples sind als 32-Bit Gleitkommazahlen kodiert.
			Die Ausgabe besteht aus Tempo $\gamma$ und Phase $\theta$ der erkannten Beatfolge
				und wird alle \num{11.61} Millisekunden aktualisiert.
			Aus dem Tempo ergibt sich die Beatperiode $T = 1 / \gamma$.
			Um die genaue Bedeutung der Phase zu verstehen,
				kann man die komplette Laufzeit des Algorithmus in $T$ lange Abschnitte unterteilen.
			So sollte sich in jedem Abschnitt ein Schlag befinden.
			Die Phase $\theta$ gibt an,
				wann innerhalb dieser Abschnitte der Schlag kommt (\SIrange{0}{360}{\degree}).
			Das ist zum Einen Hilfreich,
				um den genauen Zeitpunkt des nächsten Schlags zu bestimmen.
			Zum Anderen kann man an langsamen Drifts der Phase erkennen,
				ob das geschätzte Tempo ein bischen zu langsam oder zu schnell ist.
		}

		\subsubsection*{Funktionsweise}
		{
			% STFT und ODF
			Der Algorithmus  arbeitet ebenfalls mit einer Einsatzdetektionsfunktion (Abk.: ODF von engl. onset detection function).
			Tatsächlich verwent er sogar dieselbe komplexe Spektraldifferenzfunktion $d$ aus~\cite{2004_BeDaDuSa},
				die auch in \cite{2009_DaPlSt} verwendet wurde.
			Diese ODF bekommt einen STFT-Frame $p_{t, f}$ als Eingabe und gibt ein ODF-Sample $d_t$ zurück.
			\begin{equation}
				d_t = d(p_{t, f})
			\end{equation}
			Für die Kurzzeit-Fouriertransformation wurde ein Von-Hann-Fenster mit einer Breite von \num{1024} und eine Schrittweite von \num{512} verwendet.
			Bei einer Abtastfrequenz von \SI{44.1}{\kilo\hertz} entsteht so ein neuer STFT-Frame,
				und somit auch ein neues ODF-Sample,
				alle $512 / \SI{44.1}{\kilo\hertz} = \SI{11.61}{\milli\second}$.

			% vorverarbeitete ODF
			Im nächsten Schritt werden die ODF-Samples vorverarbeitet.
			Dazu wird zuerst der Median der Einsatzdetektionsfunktion berechnet.
			Da aus~\cite{2011_PlRoSt} nicht hervorgeht, über welchen Zeitraum der Median berechnet wird,
				wird in meiner Implementierung,
				wie in~\cite{2009_DaPlSt},
				ein \num{6} Sekunden langer Analyserahmen der Einsatzdetektionsfunktion gespeichert
				und dieser zur Berechung des Medians genutzt.
			Anders als bei~\cite{2009_DaPlSt} wird hier der Analyserahmen nicht alle \num{1.5} Sekunden,
				sondern nach jedem neuen ODF-Sample weitergeschoben,
				sodass er immer die letzten \num{6} Sekunden (\num{517} Samples) der Einsatzdetektionsfunktion enthält.
			So wird aus jedem neuen ODF-Sample $d_t$ sofort das nächste Sample $v_t$ der vorverarbeiteten ODF berechnet.
			\begin{equation}
				v_t = v(d_t) =
				\begin{cases}
					d_t & \text{falls } d_t > \text{median} \\
					0    & \text{sonst}
				\end{cases}
			\end{equation}
			Diese vorverarbeitete Einsatzdetektionsfunktion wird im nächsten Schritt genutzt,
				um die Kammfiltermatrix zu aktualisieren.

			% Beschreibung Kammfiltermatrix
			Die Kammfiltermatrix $X_{\tau, x}$ enthält eine Zeile für jede mögliche Temposchätzung, die der Algorithmus abgeben kann.
			Diese werden mit der entsprechenden Beatperiode in ODF-Sample-Dauern adressiert.
			Da der Algorithmus auf einen Tempobereich von \SIrange{80}{160}{BPM} beschränkt ist,
				ergibt sich ein minimaler Zeilenindex von
				$\tau_{min} = \lfloor(\SI{11.61}{\milli\second} \cdot \SI{160}{\per\minute})^{-1}\rfloor = 32$
				und ein maximaler Zeilenindex von
				$\tau_{max} = \lceil(\SI{11.61}{\milli\second} \cdot \SI{80}{\per\minute})^{-1}\rceil = 65$.
			Jede Zeile entählt eine Zelle für jeden diskreten Zeitpunkt innerhalb der entsprechenden Beatperiode.
			Diese haben auch jeweils einen Abstand von einer ODF-Sample-Dauer
				und werden deshalb ebenfalls in dieser Einheit adressiert.
			So ergibt sich für eine bestimmte Zeile $\tau$ ein Zellenindexbereich von $x = 0, ..., \tau - 1$.
			Der Zellenindex $x$ ist direkt mit der Phase $\theta$ korreliert,
				jedoch wird hier eine andere Variable verwendet,
				da $x$ in ODF-Sample-Dauern von $0$ bis $\tau - 1$ ist
				und $\theta$ einen Phasenwinkel in von \SIrange{0}{359}{\degree} ist
				und die Umrechnung der beiden Größen von der Beatperiode $\tau$ abhängt.

			% Kammfiltermatrixaktuelisierung
			Die Kammfiltermatrix wird mit jedem neuen vorverarbeiteten ODF-Sample $v_t$ zum Zeitpunkt $t$ aktualisiert.
			Dabei wird für jede Zeile $\tau$ die Zelle,
				die den aktuellen Zeitpunkt entspricht $x = t \bmod \tau$,
				mit der Updateregel
				\begin{equation}
					X_{\tau, x} \leftarrow
						\alpha v_t +
						(1 - \alpha) \max_{\substack{\tau_m = 32, ..., 65 \\ x_m = 0, ..., \tau - 1}}(g(\tau, \tau_m, 3.5) g(x, x_m, 6) X_{\tau, x})
				\end{equation}
				aktualisiert.
			Die Konstante $\alpha$ ist \num{0.1}
				und $g$ ist eine Gewichtsfunktion,
				die die Form einer gaußschen Normalverteilung annimmt,
				jedoch immer ein Maximum von \num{1} hat.
			$\sigma$ bestimmt also nur die Breite der Glockenkurve,
				aber nicht die Höhe,
				wie bei einer gewöhnlichen Normalverteilung.
			\begin{equation}
				g(c, m, \sigma) = e^{-\frac{(c - m)^2}{2\sigma^2}}
			\end{equation}

			% Idee dahinter
			Die Idee hinter dieser Updateregel ist,
				dass sich,
				in der zum richtigen Tempo gehörenden Zeile,
				die Peaks der vorverarbeiteten ODF immer wieder an der gleichen Stelle überlagern
				und somit eine Zelle in dieser Zeile einen sehr hohen Wert annimmt,
				während in anderen Zeilen,
				wegen des unpassenden Tempos dieser Zeilen,
				die Peaks immer auf eine andere Stelle treffen
				und es so nicht schaffen eine einzelne Zelle hochzudrücken.
			Um das geschätzte Tempo und die Phase zu bestimmen,
				kann man nach dem Maximum in der Matrix suchen.
			Bevor das geschiet,
				durchläuft diese jedoch erst einen weiteren Verarbeitungsschritt.

			% Y-Matrix
			Zunächst wird jede Zeile mit einem Rayleigh-Gewicht multipliziert.
			Diese Gewichtsfunktion hat ihr Maximum bei \SI{120}{BPM} und flacht nach außen hin ab.
			So werden häufiger vorkommende Tempi, bevorzugt.
			Die Rayleigh-Funktion ist wie folgt definiert:
			\begin{equation}
				r(\tau) = \frac{\tau}{\beta^2}e^{\frac{-\tau^2}{2\beta^2}}
			\end{equation}
			wobei die Konstante $\beta$ den höchsten Punkt der Gewichtsfunktion bestimmt.
			Ein Wert von $\beta = 43$ ODF-Samples entspricht hier einem Tempo von
				$(43 \cdot \SI{11.61}{\milli\second})^{-1} = \SI{120}{\per\minute}$.
			Die Zeile des rightigen Tempos hat wahrscheinlich eine starke Konzentration an hohen Werten um den Punkt der richtigen Phase herum.
			Das bedeutet,
				dass diese Zeile eine niedrigere Entropie hat als andere Zeilen,
				deren Werte gleichmäßiger über die gesamte Zeile verteilt sind.
			Die Entropie einer Matrixzeile ist durch
				\begin{equation}
					h(\tau) = \sum_{x = 0}^{\tau - 1} -p(\tau, x) log(p(\tau, x))
				\end{equation}
				definiert, wobei
				\begin{equation}
					p(\tau, x) = \frac{X_{\tau, x}}{\sum_{x_i = 0}^{\tau - 1}X_{\tau, x_i}}
				\end{equation}
				als Häufigkeitsverteilung einer Zeile interpretiert werden kann.
			Als nächstes wird in der resultierenden Matrix
				\begin{equation}
					Y_{\tau, x} = \frac{r(\tau)}{h(\tau)}X_{\tau, x}
				\end{equation}
				nach dem Maximum gesucht.
			Der Matrixindex $(\tau_{\text{neu}}, x_{\text{neu}})$ des Maximums gibt das wahrscheinlichste Tempo und die wahrscheinlichste Phase an.

			% Tempoübergänge
			Da das so ermittelte Tempo-Phase-Paar  z. B. bei Unregelmäßigkeiten in der Musik etwas hin und her springen kann,
				gibt es noch eine Tempoupdateregel,
				die dafür sorgt,
				dass sich das Tempo und die Phase nur dann ändern können,
				wenn entweder die neue Hypothese sich in der Matrix relativ nah an der alten befindet,
				oder sie viel stärker ist als die alte.
			Hierfür definieren wir eine Gewichtsfunktion,
				welche Werte in der Nähe der aktuellen Hypothese $(\tau_f, x_f)$ bevorzugt.
			\begin{equation}
				w(\tau_{\text{neu}}, x_{\text{neu}}) =
					g(\tau_f, \tau_{\text{neu}}, 4)
					g(x_f, x_{\text{neu}}, 10)
			\end{equation}
			Die Standartabweichungen von \num{4} und \num{10} wurden durch~\cite{2011_PlRoSt} experimentell ermittelt.
			Die aktuelle Hypothese wird mit folgender Updateregel aktuelisiert:
			\begin{equation}
				(\tau_f, x_f) \leftarrow
				\begin{cases}
					(\tau_{\text{neu}}, x_{\text{neu}}) &
						\text{falls } w(\tau_{\text{neu}}, x_{\text{neu}}) Y_{\tau_{\text{neu}}, x_{\text{neu}}} > Y_{\tau_f, x_f} \\
					(\tau_f, x_f) &
						\text{sonst}
				\end{cases}
			\end{equation}

			% Schlusssatz
			All diese Berechnungen werden für jedes neue Sample der Einsatzdetektionsfunktion durchgeführt.
			So gibt der Algorithmus alle \SI{11.61}{\milli\second} eine neue Tempo-Phase-Hypothese aus.
		}
	}
}
