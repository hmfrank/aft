\chapter{Einleitung}
\label{einleitung}
\acresetall

\section{Kontext}
{
	% leicht für Menschen, schwer für Maschinen
	Die Erkennung der regelmä{\ss}igen Schläge des Taktes eines Musikstücks ist eine triviale Aufgabe für Menschen.
	Manchmal tippen wir sogar unterbewusst unseren Fu{\ss} zum Takt der Musik,
		ohne aktiv darüber nachzudenken.
	Es ist jedoch sehr schwer diese Aufgabe zu automatisieren,
		da Maschinen kein Taktgefühl haben.
	Dennoch gibt es verschiedene Algorithmen,
		die genau das bewerkstelligen.
	% Anwendungen
	Diese Beaterkennungsalgorithmen werden z. B. in Videobearbeitungssoftware verwendet,
		um die visuelle Spur automatisch mit der Audiospur zu synchronisieren.
	Deshalb werden besonders beim Erstellen von Musikvideos solche Algorithmen eingesetzt.
	Auch in Audiobearbeitungs- und Aufnahmesoftware finden Beaterkennungsalgorithmen Einsatz,
		z. B. bei der automatischen Indizierung von Musikstücken.
	Bei Liveauftritten spielen Beaterkennungsalgorithmen auch eine wichtige Rolle.
	Hier werden sie eingesetzt um z. B. die Bühnenbeleuchtung automatisch und passend zur Musik zu steuern.
}

\section{Zielsetzung}
{
	Ziel dieser Arbeit ist es,
		einen umfangreichen, qualitativen Vergleich von ausgewählten Beaterkennungsalgorithmen zu erstellen,
		der hilfreich für die Auswahl eines geeigneten Algorithmus für einen Anwendungsfall ist.
	Dazu wurden drei Algorithmen analysiert und implementiert
		und mehrere Vergleichstests erstellt.
	Die Tests lassen die Algorithmen auf mit Beatpositionen annotierten Musikstücken laufen,
		beobachten dabei deren Ausgabe
		und vergleichen sie mit den Annotationen der Lieder.
	Auf diese Art und Weise werden mehrere Metriken zur Bewertung der Algorithmen erstellt,
		anhand welcher die Algorithmen verglichen werden.
}

\section{Kapitelübersicht}
{
	% TODO: schreiben, wenn alle anderen Kapitel komplett sind
}

