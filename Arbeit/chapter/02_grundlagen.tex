\chapter{Grundlagen und Begriffe}
\label{grundlagen}

\section{Musikalische Grundbegriffe}
{
	Um die in dieser Arbeit untersuchten Beaterkennungsalgorithmen zu verstehen,
		wird etwas musikalisches Grundwissen benötigt.
	In diesem Abschnitt werden deshalbt Grundbegriffe erklärt,
		die in der restlichen Arbeit verwendet werden.

	\paragraph{Takt}
	{
		Fast jedes Musikstück ist zeitlich in kurze und gleichlange Abschnitte unterteilt.
		Diese Abschnitte nennt man Takte
			und sie geben der Musik eine gewisse Struktur,
			da sich oft bestimmte Muster, wie z. B. Schlagzeugrhythmen, in jedem oder jedem zweiten Takt wiederholen
			und auch die Grenzen von Strophe, Refrain usw. an Taktgrenzen orientiert sind.
		Die Länge eines Taktes hängt von der Taktart und dem Tempo des Stücks ab
			und ist meist im einstelligen Sekundenbereich.
	}

	\paragraph{Schlag (Beat)}
	{
		Ein Schlag (engl. Beat) ist eine Unterteilung des Taktes.
		Jeder Takt hat genau gleichviele Schläge (au{\ss}er es gibt einen Taktartwechsel im Stück).
		Die Abfolge von Schlägen in einem Lied ist oft dieselbe,
			die ein Mensch beim Hören der Musik intuitiv mit dem Fu{\ss} mitklopfen würde.
	}

	\paragraph{Taktart}
	{
		Die Taktart besteht aus zwei Teilen
			und wird deshalb oft als ungekürzter Bruch angegeben (z. B. 4/4).
		Der Zähler gibt an,
			aus wie vielen Schlägen ein Takt besteht.
		Der Nenner gibt an,
			wie lange ein Schlag dauert
			und wird nicht in einer Zeiteinheit,
			sondern als Notenwert angegeben.
		Ein Nenner von 4 bedeutet z. B.,
			dass ein Schlag so lange wie eine Viertelnote dauert.
		Das ist für die musikalische Notation relevant,
			hat aber keine Bedeutung für diese Arbeit,
			da dahinter kein Bezug zur tatsächlichen Dauer (in Sekunden) eines Schlages steckt.
		Die meisten Lieder haben einen 4/4-Takt,
			weshalb manche Algorithmen in dieser Arbeit speziell auf diese Taktart ausgelegt sind.
	}

	\paragraph{Tempo}
	{
		Das Tempo eines Musikstückes gibt an,
			wie schnell es gespielt wird
			und bestimmt somit die tatsächliche Dauer (in Sekunden) eines Schlages.
			und in BPM (engl. beats per minute) angegeben.
		Üblicherweise befinden sich Tempi von Musikstücken in einem Bereich zwischen 60 und 180 BPM.
	}
}

\section{Grundlagen der Beaterkennung}
{
	Beaterkennung ist die Aufgabe,
		aus rohen Audiodaten die Anfangszeitpunkte der einzelnen Schläge eines Musikstücks zu extrahieren.
	Zusätzlich wird von manchen Algorithmen auch das Tempo berechnet.

	\paragraph{}
	{
		Das Beaterkennungsproblem bringt folgende Herausforderungen mit sich:
	}

	\paragraph{1} % kein eindeutiges Beatgeräusch
	{
		Es gibt nicht immer ein bestimmtes Geräusch,
			dass die Anfangszeitpunkte der Schläge markiert.
		Es kann sogar sein, dass auf einen Beat kein Signal kommt.
	}

	\paragraph{2} % Onsetungenauigkeit
	{
		Da oft viele Instrumente, mit sehr unterschiedlichen Klängen, in einem Lied spielen,
			ist es schwierig,
			besonders genaue Einsatzzeitpunkte von Noten zu bestimmen.
	}

	\paragraph{3} % Echtzeit
	{
		Bei der Beaterkennung in Echtzeit können nur Samples aus der Vergangenheit genutzt werden,
			wohingegen bei nicht-Echtzeit-Algorithmen alle Samples verwendet werden können.
	}

	\paragraph{4} % nicht trivial
	{
		Es ist keine effektife Methode die Lautstärkemaxima eines Audiosignals zur Beaterkennung zu nutzen,
			da es auch viele Peaks gibt,
			die nicht auf einen Beat passen.
		% TODO: peak glossar
		% TODO: vlt auch Glossar für alle Abkürzungen
	}

	\paragraph{5}  % Nichteindeutigkeit
	{
		Die Zuordnung von Audiodaten zu Beatzeitpunkten und Tempo ist nicht eindeutig.
		Schläge und Takte sind lediglich eine gedankliche Einteilung des Stücks,
			die man nicht direkt hören kann,
			sondern sich nur dadurch bemerkbar machen,
			dass musikalische Ereignisse
			(z. B. Anfang und Ende einer Note, Schlagzeugtöne oder Harmoniewechsel)
			an deren Grenzen orientiert sind.
		Man könnte also z. B. ein Stück mit 160 BPM auch als eines mit 80 BPM betrachten
			(angenommen der Zähler der Taktart ist durch zwei teilbar),
			indem man jeweils zwei Takte als einen interpretiert.
		Das Audiosignal wäre bei beiden Varianten das gleiche.
	}


	% difficulties
	% * viele instrumente --> schwierig genaue Onsets zu bestimmen
	% * man kann nicht einfach den Peak picken um beats zu finden, weil da es auch viele Energy-Peaks an Stellen wo kein Beat is gibt
	% * ...
	% * ...
}
