\chapter{Grundlagen und Begriffe}
\label{grundlagen}

\section{Musikalische Grundbegriffe}
{
	Um die in dieser Arbeit untersuchten Beaterkennungsalgorithmen zu verstehen,
		wird etwas musikalisches Grundwissen benötigt.
	In diesem Abschnitt werden deshalbt Grundbegriffe erklärt,
		die in der restlichen Arbeit verwendet werden.

	\paragraph{Takt}
	{
		Fast jedes Musikstück ist zeitlich in kurze und gleichlange Abschnitte unterteilt.
		Diese Abschnitte nennt man Takte
			und sie geben der Musik eine gewisse Struktur,
			da sich oft bestimmte Muster, wie z.B. Schlagzeugrhythmen, in jedem oder jedem zweiten Takt wiederholen
			und auch die Grenzen von Strophe, Refrain usw. an Taktgrenzen orientiert sind.
		Die Länge eines Taktes hängt von der Taktart und dem Tempo des Stücks ab
			und ist meist im einstelligen Sekundenbereich.
	}

	\paragraph{Schlag (Beat)}
	{
		Ein Schlag (engl. Beat) ist eine Unterteilung des Taktes.
		Jeder Takt hat genau gleichviele Schläge (außer es gibt einen Taktartwechsel im Stück).
		Die Abfolge von Schlägen in einem Lied ist oft dieselbe,
			die ein Mensch beim Hören der Musik intuitiv mit dem Fuß mitklopfen würde.
	}

	\paragraph{Taktart}
	{
		Die Taktart besteht aus zwei Teilen
			und wird deshalb oft als ungekürzter Bruch angegeben (z. B. 4/4).
		Der Zähler gibt an aus wievielen Schlägen ein Takt besteht.
		Der Nenner gibt an wie lange ein Schlag dauert,
			jedoch nicht in einer Zeiteinheit sondern als Notenwert.
		Ein Nenner von 4 bedeutet z. B., dass ein Schlag so lange wie eine Viertelnote dauert.
		Das ist für die musikalische Notation relevant,
			hat aber keine Bedeutung für diese Arbeit,
			da dahinter kein Bezug zur tatsächen Dauer (in Sekunden) eines Schlages steckt.
		Die meisten Lieder haben einen 4/4-Takt,
			weshalb manche Algorithmen in dieser Arbeit speziell auf diese Taktart ausgelegt sind.
			% TODO: sagen welche Algorithmen genau
	}

	\paragraph{Tempo}
	{
		Das Tempo eines Musikstückes gibt an wie schnell es gespielt wird
			und bestimmt somit die tatsächliche Dauer (in Sekunden) eines Schlages.
		Es wird in BPM (beats per minute) angegeben.
		Üblicherweise befinden sich Tempi von Musikstücken in einem Bereich zwischen 60 und 180 BPM.
	}
}

\section{Beaterkennung}
{
	Beaterkennung ist die Aufgabe aus rohen Audiodaten die Anfangszeitpunkte der einzelnen Schläge eines Musikstücks zu extrahieren.
	Zusätzlich wird von manchen Algorithmen auch das Tempo berechnet.

	Ein Problem dabei ist,
		dass die Zuordnung von Audiodaten zu Beatzeitpunkten und Tempo nicht eindeutig ist.
	Schläge und Takte sind lediglich eine gedankliche Einteilung des Stücks,
		die man nicht direkt hören kann
		sondern sich nur dadurch bemerkbar macht,
		dass musikalische Ereignisse
		(z. B. Anfang und Ende einer Note, Schlagzeugtöne oder Harmoniewechsel)
		an deren Grenzen orientiert sind.
	Man könnte also z. B. ein Stück mit 160 BPM auch als eines mit 80 BPM betrachten
		(angenommen der Zähler der Taktart ist durch zwei teilbar),
		indem man jeweils zwei Takte als einen interpretiert.
	Das Audiosignal ist bei beiden Varianten das selbe.
}
