\chapter{Zusammenfassung}
\label{zusammenfassung}

\section{Fazit}
{
	Im Rahmen dieser Arbeit sollte eine Auswahl von Beaterkennungsalgorithmen implementiert
		und mit Hilfe von verschiedenen Tests verglichen werden.
	Es wurden die Algorithmen \cite{2001_BeatThis}, \cite{2009_DaPlSt} und \cite{2011_PlRoSt} ausgewählt und implementiert.
	Vier Tests wurden entworfen und implementiert,
		die die Algorithmen auf Genauigkeit der Tempovorhersage, Genauigkeit der Beatzeitpunkte, längste korrekte Beatfolge und Rechenzeit testen,
		wobei \cite{2001_BeatThis} nur auf Genauigkeit der Tempovorhersage und auf Rechenzeit getestet wurde,
		weil dieser Algorithmus keine Beatzeitpunkte ausgibt.
	Für den Test wurde der Datensatz von~\cite{2012_HoDaZaOlGo} verwendet,
		welcher für Beaterkennungsalgorithmen besonders schwierige Lieder enthält.
	Der Algorithmus von~\cite{2009_DaPlSt} schnitt beim Test auf Genauigkeit der Tempovorhersage am besten ab.
	Die anderen beiden Algorithmen schätzten bei einem Gro{\ss}teil der Lieder das Tempo zu schnell ein.
	Das verursachte bei~\cite{2011_PlRoSt},
		dass der Algorithmus zu viele Beats ausgab,
		weshalb auch die Tests auf Genauigkeit der Beatzeitpunkte und längste korrekte Beatfolge bei \cite{2009_DaPlSt} besser ausfielen.
	Auch beim Test auf Rechenzeit,
		schnitt~\cite{2009_DaPlSt} am besten ab,
		gefolgt von \cite{2001_BeatThis} und \cite{2011_PlRoSt}.
	Die Unterschiede in der Rechenzeit sind hauptsächlich auf die unterschiedlichen Datenmengen,
		die die Algorithmen intern verarbeiten,
		und die unterschiedlichen Aktualisierungsraten der Tempo- und Beatzeitpunktvorhersage zurückzuführen.
}

\section{Ausblick}
{
	% leichteren Datensatz nehmen
	Aufgrund der relativ schlecht ausgefallenen Testergebnisse,
		besteht die Möglichkeit,
		dass in vielen Liedern die Algorithmen gar keinen Takt finden konnten,
		weil der verwendete Datensatz zu schwer war.
	Würde man in zukünftigen Arbeiten die Algorithmen mit einem leichteren Datensatz testen,
		könnte man eventuell feinere Leistungsunterschiede feststellen.

	% mehr Sachen testen
	Die Messung von anderen Metriken kann auch weitere hilfreiche Informationen für die Auswahl eines Beaterkennungsalgorithmus für einen bestimmten Anwendungfall bringen.
	Eine Trennung der Testergebnisse nach Musikrichtung z. B. könnte Aufschlüsse über die Eignung der Algorithmen für bestimmte Genres geben,
		welche wiederum für die Auswahl eines Algorithmus für die Lichtsteuerung eines Live-Auftritts hilfreich sein kann.
	Auch Metriken wie die Einlaufzeit,
		also die Zeit,
		die ein Algorithmus braucht um vom Kaltstart auf das richtige Ergebniss zu kommen,
		oder die Eignung eines Algorithmus für bestimmte Taktarten,
		können hilfreiche Informationen für die Benutzer dieser Algorithmen hervorbringen.

	% krassere Algorithmen testen
	Zukünftige Forschungen könnten au{\ss}erdem weitere state-of-the-art Beaterkennungsalgorithmen vergleichen,
		wie zum Beispiel~\cite{2000_Di} und \cite{2001_Go},
		die für diese Arbeit aufgrund fehlender Informationen nicht implementiert werden konnten,
		oder auch neuere Algorithmen.
	Manche Algorithmen geben speziellere Metainformationen zu den einzelnen Beats aus,
		wie zum Beispiel der Algorithmus von~\cite{2001_Go},
		welcher für jedern Beat bestimmt,
		auf welchem der vier Hauptschläge des 4/4-Takts sich der Beat befindet.
	Solche Algorithmen bieten wiederum eine Grundlage für speziellere Tests.
}
