\chapter{Verwandte Arbeiten}
\label{verwandte_arbeiten}
\acresetall

In diesem Kapitel werden alle Paper,
	die im Rahmen dieses Forschungsprojekts verwendet wurden,
	kurz vorgestellt.

% 1997 Goto, Muraoka - Issues in Evaluating Beat Tracking Systems
\paragraph{\cite{1997_GoMu1}}
{
	befasst sich mit Problemen bei der Bewertung von Beaterkennungsalgorithmen.
	Es wird eine Methode zur Evaluierung der Genauigkeit dieser Algorithmen vorgestellt,
		welche auch typische Fehler,
		die solche Systeme machen,
		in Betracht zieht.
	Diese Methode wird auch in dieser Arbeit verwendet.
}

% __________ ALGORITHMEN ______________________________________________________

% 2000 Dixon
\paragraph{\cite{2000_Di}}
{
	stellt einen vielversprechenden Algorithmus zur Erkennung von Beats vor,
		der auf der Häufigkeitsverteilung von Zeitintervallen zwischen den Beats basiert.
}

% 2001 Beat This
\paragraph{\cite{2001_BeatThis}}
{
	stellen einen Algorithmus vor,
		der das Tempo eines Musikstücks erkennt.
	Dieser wurde nicht für Echtzeitanwendungen konzipiert
		aber kann trotzdem dafür verwendet werden.
	Er basiert auf einer Kammfiltermatrix
		und hat generell einen relativ einfachen Aufbau im Vergleich zu den anderen Algorithmen in dieser Arbeit.
}

% 2001 Goto
\paragraph{\cite{2001_Go}}
{
	beschreibt ein komplexes Beaterkennungssystem,
		welches nicht nur Beatpositionen erkennt,
		sondern auch die hierarchische Taktstruktur eines 4/4-Taktes,
		welche aus Takten, halben Takten und viertlen Takten (Schlägen) besteht.
	Das System ist eine Kombination und Verbesserung zweier älterer Systeme,
		die zum einen für Musik mit Schlagzeug (\cite{1994_GoMu}, \cite{1995_GoMu1}, \cite{1998_GoMu})
		und zum anderen für Musik ohne Schlagzeug (\cite{1996_GoMu}, \cite{1997_GoMu2}, \cite{1999_GoMu})
		ausgelegt waren.
	Das neue System funktioniert sowohl mit Musik mit Schlagzeug als auch mit Musik ohne Schlagzeug.
	Au{\ss}erdem ist es vielversprechend,
		weil mehrere verschiedene Informationen verwendet werden,
		um die Ausgabe zu verbessern,
		namentlich die Einsatzzeitpunkte, Harmoniewechsel und gängige Schlagzeugmuster.
}

\paragraph{\cite{2004_BeDaDuSa}}
{
	vergleicht verschiedene \acp{ODF},
		welche wesentlicher Bestandteil vieler Beaterkennungsalgorithmen sind.
	Au{\ss}erdem wird eine \ac{ODF} vorgestellt,
		die sowohl auf Lautstärkeänderungen
		als auch auf Phasenänderungen in der komplexen Domäne
		reagiert.
	Diese \ac{ODF} wird auch in zwei der in dieser Arbeit verglichenen Algorithmen verwendet.
}

\paragraph{\cite{2009_DaPlSt}}
{
	präsentiert ein Beaterkennungssystem,
		das ein Hybrid aus zwei Vorgängern ist.
	So wurde die Beatzeitpunktbestimmung von~\cite{2007_El}
		und die Tempoerkennung von~\cite{2007_DaPl} inspiriert.
}

% 2011 PlRoSt
\paragraph{\cite{2011_PlRoSt}}
{
	erklärt einen echtzeitfähigen Beaterkennungsalgorithmus mit visuellem Interface.
	Hierbei wird eine intuitive Repräsentation einer Kammfiltermatrix visualisiert,
		mit der der Benutzer interagieren kann.
	Der Algorithmus kann auch ohne Nutzerinteraktion laufen,
		aber könnte dann Schwierigkeiten haben,
		die Mehrdeutigkeit des Beaterkennungsproblems aufzulösen.
}

% __________ DATENSATZ ________________________________________________________

\paragraph{\cite{2012_HoDaZaOlGo}}
{
	stellt eine Methode vor zur Identifikation von komplexer Musik,
		die für Beaterkennungsalgorithmen schwierig ist.
	Diese Methode wird benutzt,
		um einen schwierigen Datensatz für die Evaluation von Beaterkennungsalgorithmen zu erstellen.
	Es wird auch beschrieben,
		wie die einzelnen Lieder des Datensatzes annotiert wurden.
	Der so entstandene Datensatz wird auch in den Vergleichstests dieser Arbeit genutzt.
}
